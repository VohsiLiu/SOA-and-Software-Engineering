    \begin{table}[H]
        \centering
        \resizebox{\textwidth}{!}{
        \begin{tabular}{|c|l|}
        \hline
        \textbf{集成层}                                                   & \begin{tabular}[c]{@{}l@{}}SOA解决方案中的关键支持部件,用以在服务请求者和服务提供者之间,完成服务请求的 \\ 中介、路由和转换。\end{tabular}                                                 \\ \hline
        \textbf{\begin{tabular}[c]{@{}c@{}}服务质量层\\ (QoS)\end{tabular}} & \begin{tabular}[c]{@{}l@{}}从各个方面(可用性、可靠性、安全性等非功能性需求)提供解决方案层级的QoS管理。\\ 服务质量层不关注于服务层级的QoS控制,而是着眼于为解决方案层级的 QoS 控制提供 \\ 支持、跟踪、监视和管理。\end{tabular} \\ \hline
        \textbf{数据架构层}                                                 & \begin{tabular}[c]{@{}l@{}}为了方便值链集成(集成来源于不同开发方的服务),数据架构层为领域相关的数据架构提 \\ 供统一的表达和支持机制。\end{tabular}                                             \\ \hline
        \textbf{治理层}                                                   & \begin{tabular}[c]{@{}l@{}}提供用以确保 SOA 解决方案的设计原则;通常使用最佳实践的方式,来提供如何在各个层 \\ 次中构建 SOA 解决方案的原则、如何监管运营中的系统,并在运行时处理异常的原则。\end{tabular}               \\ \hline
        \end{tabular}}
    \end{table}