\begin{longtable}{|m{3.7cm}<{\centering}|m{11cm}|}
    \hline
    \textbf{element} & \multicolumn{1}{c|}{\textbf{描述}} \\ \hline
    partners &
    \vspace{-1.3em}
    \begin{itemize}[leftmargin=1.5em,itemsep=-2pt]
        \item 定义正在构建的业务流程与其合作流程之间的关系
        \item 合作伙伴可以作为
        \vspace{-0.2em}
        \begin{itemize}[leftmargin=1.5em,itemsep=-2pt]
            \item 提供业务流程所使用的服务的消费者
            \item 提供被业务流程使用的服务的提供者
            \item 激活业务流程的服务
        \end{itemize}
    \vspace{-1.5em}
    \end{itemize}                                           
    \\ \hline
    partner link types &
    \vspace{-1.3em}
    \begin{itemize}[leftmargin=1.5em,itemsep=-2pt]
        \item 定义两个Web服务之间的对话关系
        \item 在WSDL中定义,并给出与portType相关联的角色
    \vspace{-1.5em}
    \end{itemize}                                           
    \\ \hline
    partner links &
    \vspace{-1.3em}
    \begin{itemize}[leftmargin=1.5em,itemsep=-2pt]
        \item 对业务流程与其交互的合作伙伴服务进行建模
        \item 由partner link type所特征化
        \item 定义业务流程内的静态关系
    \vspace{-1.5em}
    \end{itemize}                                           
    \\ \hline
    business partners &
    \vspace{-1.3em}
    \begin{itemize}[leftmargin=1.5em,itemsep=-2pt]
        \item 使用partner元素对需要多个对话关系的业务合作伙伴关系进行建模
        \item 允许对partner links进行分组
        \item 禁止重叠
    \vspace{-1.5em}
    \end{itemize}                                           
    \\ \hline
    endpoint references &
    \vspace{-1.3em}
    \begin{itemize}[leftmargin=1.5em,itemsep=-2pt]
        \item 为服务提供端口特定数据提供动态绑定机制
        \item 允许将服务提供者与partner links中的角色进行动态绑定
    \vspace{-1.5em}
    \end{itemize}                                           
    \\ \hline
    activities &
    \vspace{-1.3em}
    \begin{itemize}[leftmargin=1.5em,itemsep=-2pt]
        \item 基本活动,例如接收、回复和调用
        \item 结构化活动
        \vspace{-0.2em}
        \begin{itemize}[leftmargin=1.5em,itemsep=-2pt]
            \item 按顺序执行(序列、开关和循环)
            \item 并发执行(流程)
            \item 非确定性选择(pick)
        \end{itemize}
        \item 属性,名称,连接条件,源,目标
    \vspace{-1.5em}
    \end{itemize}                                           
    \\ \hline
    data handling &
    \vspace{-1.3em}
    \begin{itemize}[leftmargin=1.5em,itemsep=-2pt]
        \item 在业务流程中建模状态
        \item 三种处理方式
        \vspace{-1em}
        \begin{multicols}{3}
        \begin{itemize}
            \item 变量
            \item 表达式
            \item 分配
        \end{itemize}
        \end{multicols}
        \vspace{-1.2em}
        \item 提供XML数据类型和WSDL消息类型
    \vspace{-1.5em}
    \end{itemize}                                           
    \\ \hline
    Correlation &
    \vspace{-1.3em}
    \begin{itemize}[leftmargin=1.5em,itemsep=-2pt]
        \item 用于处理进程之间长时间的有状态对话
        \item 用于识别新的业务流程
    \vspace{-1.5em}
    \end{itemize}                                           
    \\ \hline
    Scope &
    \vspace{-1.3em}
    \begin{itemize}[leftmargin=1.5em,itemsep=-2pt]
        \item 每个活动行为的上下文
        \item 共有五种:故障处理程序、事件处理程序、补偿处理程序、数据变量和相关性集
    \vspace{-1.5em}
    \end{itemize}                                           
    \\ \hline
\end{longtable}