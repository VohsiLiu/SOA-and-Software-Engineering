\begin{table}[H]
    \centering
    \begin{tabular}{|c|l|}
    \hline
    \textbf{服务层}   & 服务层作为前后台连通的接口,功能同上。   \\ \hline
    \textbf{业务过程层} & \begin{tabular}[c]{@{}l@{}}以组合和分解的方式来处理业务逻辑。\\ 从组合角度出发,业务过程层使用服务层来快速组合服务,并协调业务过程来满足消\\ 费者需求;从分解的角度出发,业务过程层将业务需求分解为能够由概念上的服务簇\\ 所表达的任务。\\ 业务服务层着眼于从协作和管理一些列过程的角度出发,采用也无流程来构建SOA\\解决方案。\\ 存在两种组合方式:编排和编导(二者功能上等价,主流模式为编排)。\end{tabular} \\ \hline
    \textbf{消费者层}  & \begin{tabular}[c]{@{}l@{}}消费者层负责表达对业务过程层、服务层及其他层次的调用。\\ 通过为业务服务快速构建用户接口来满足消费者的需求。\\ 消费者层负责构建SOA解决方案与用户之间进行交互的前端接口。\\ 消费者层可能需要同时支持不同种类的用户和渠道。\\ 为了提升展现性能,往往需要支持缓存机制。\end{tabular}                                                         \\ \hline
    \end{tabular}
\end{table}