\vspace{-0.5em}
\begin{spacing}{1.2}
    \centering
    \begin{longtable}{|m{1.8cm}<{\centering}|m{6.5cm}|m{6.5cm}|}
		\hline
    \textbf{特点} & \multicolumn{1}{c|}{\textbf{面向对象的计算}}                   & \multicolumn{1}{c|}{\textbf{面向服务计算}}                                                          \\ \hline
    方法论         & 通过定义紧耦合的类来进行应用开发;应用架构为基于继承关系的层次式架构;从构造函数——通过类或模型——到系统设计 & 通过定义松耦合的服务来进行应用开发,并将服务组装成可执行的应用;从系统模型到服务模块,从服务抽象定义到服务实现绑定;通过搜索获得可用的服务实现                       \\ \hline
    抽象和协作层次     & 往往由一个团队来负责应用的开发,并负责整个生命周期;开发者必须了解应用领域知识和编程              & 开发任务由三个独立方承担:应用程序开发者,服务提供方和服务代理;其中,应用程序开发者需要了解应用逻辑,但不需要了解具体的服务是如何实现的;服务提供者需要编程能力,但不必了解使用服务的应用 \\ \hline
    代码共享和复用     & 代码复用通过类成员的继承和库函数加以实现。其中库函数在编译时引入,且往往是平台相关的              & 代码在服务层次复用。服务使用标准的结构,并发布在Internet库中。服务是平台无关的,且能够被查找并远程调用。服务代理支持系统的服务共享                         \\ \hline
    动态绑定和重新组合   & 在运行时将名称和方法进行关联。方法必须在应用部署前链接到可执行的代码                      & 在运行时将服务调用和服务进行绑定。可以在应用部署后,再进行服务选定。这一特色使得应用可以在运行时重组                                            \\ \hline
    重组          & 多在设计时决定导入的组件                                            & 可以动态改变应用系统中服务的组合关系,以及服务定义与服务实现之间的绑定关系,即实现动态地添加、修改、删除各个服务节点                                    \\ \hline
    组件通讯和接口     & 与平台和语言有关,例如C++程序难以直接和Java程序通信                           & 与平台和语言无关。组件间通过标准协议通信,如XML,WSDL和SOAP                                                           \\ \hline
    系统维护        & 用户需要时常升级软件,且在执行升级时,应用必须停止                               & 通过互联网升级系统,因为服务多运行在远程服务器上,用户通过互联网进行访问。维护对用户透明                                                  \\ \hline
    可靠性         & 在设计时决定可靠性的方法                                            & 对于服务提供者,每个服务相对简单,更加可靠。对于应用程序存在多个满足同一需求的服务,可用过将故障服务的节点断开并重新绑定到备选服务节点上,获得不间断的应用系统               \\ \hline
    软件拥有        & 软件作为产品销售,为用户所拥有                                         & 软件存在并执行于独立的服务提供商的设备上,用户按照每次对服务使用付费,而不是按照软件产品付费                                                \\ \hline
    \end{longtable}
	\end{spacing}
\vspace{-0.5em}
