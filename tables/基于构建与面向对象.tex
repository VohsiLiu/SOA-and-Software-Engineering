\begin{table}[H]
    \centering
    \begin{tabular}{|c|c|c|}
    \hline
                                                          & \textbf{基于构件}                                                                            & \textbf{面向对象}                                                                             \\ \hline
    抽象视角                                                  & \begin{tabular}[c]{@{}c@{}}构件是对客观世界的实体或者实体\\ 联合能提供的功能和服务的建模;\\ 仅仅关注实体的功能和服务\end{tabular} & \begin{tabular}[c]{@{}c@{}}对象是对客观世界基本实体的抽象,\\ 强调对实体的对应及对实体的建模;\\ 涉及实体的静态属性特征\end{tabular} \\ \hline
    \begin{tabular}[c]{@{}c@{}}可复用程度\\ 和复用机制\end{tabular} & 以组合的方式实现复用                                                                               & 以继承的方式实现复用                                                                                \\ \hline
    粒度不同                                                  & 大                                                                                        & 小                                                                                         \\ \hline
    \end{tabular}
\end{table}