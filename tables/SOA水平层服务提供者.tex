\begin{table}[H]
    \centering
    \vspace{-0.5em}
    \begin{tabular}{|c|l|}
        \hline
        \textbf{操作型系统层} & \begin{tabular}[c]{@{}l@{}}包括ISV(独立软件开发商)提供的打包应用、客户应用、遗留系统等。\\ 该层的应用(不一定面向服务)往往只为一个目的、服务于一类特定用户。\end{tabular}                                                                                                                                         \\ \hline
        \textbf{服务组件层}  & \begin{tabular}[c]{@{}l@{}}包括用于提供用以实现服务层中所定义服务的代码容器,其中一个服务组件依赖于\\ 操作系统型层次中的一些打包组件、服务层中的一些服务、业务过程层中的一些业\\ 务过程。\\ 该层可能实现多个方法,但其中只有一部分会被服务层封装为服务。\\ 从调用角度出发,服务组件层负责完成输入转换和输出配置的自动化逻辑。\end{tabular}                                                       \\ \hline
        \textbf{服务层}    & \begin{tabular}[c]{@{}l@{}}将SOA三角操作模型扩展为综合的逻辑层次,以支持服务注册、服务分解、服务\\ 发现、服务绑定、接口聚合和生命周期管理。服务层负责定位合适的服务提供者,\\ 并绑定到具体目标服务接口;同时负责以服务组合的形式封装服务对外提供。\\ \textbf{服务簇}是服务层中的核心概念,是一类从概念上服务于同一业务功能的服务集合。\\ 服务簇中的服务可以由不同的功能提供者所发布,并在具体的特性上有所差异(但\\ 都能满足业务功能需求)。\end{tabular} \\ \hline
    \end{tabular}
\end{table}